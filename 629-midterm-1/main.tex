\documentclass[12pt]{article}
\usepackage[margin=1in]{geometry}

\usepackage{setspace}
\doublespacing

\title{Quantitative research is best research}
\author{Colby Goettel}

\begin{document}
\maketitle

% Prompt: What are the strengths and limitations of (or problems with) the positivist-quantitative approach to educational research, including some specific techniques? From your perspective, when is quantitative research advisable? Be sure to support claims and defend your position.

% DEFINE
Quantitative research is reliable, objective, hard. It is controlled. Systematic. Quantitative research is performed by observing the world, then using mathematical and statistical techniques to analyze the results. Its strength lies in the fact that it provides hard evidence to support claims.

% STRENGTHS, INCLUDING TECHNIQUES
When test subjects are randomly selected from the population, it's possible to infer that the results are representative of the population. When the treatments are randomly assigned, causal relationships can be inferred. These characteristics compound and lend power to quantitative studies: when test subjects are randomly selected and treatments are randomly assigned, inferences to populations \emph{and} causal relationships can be inferred.

Another great strength of quantitative research is that it is systematic. When performing experiments, researchers can go through related variables, in order, and systematically check things off as they relate or don't relate. This principle allows researchers to remove bias and confounding variables. The research becomes purer because the methodology is sound, following the principles of the scientific method.

An excellent case study of the importance of quantitative research is the Monty Hall problem. This problem has confused computer science students for decades because the correct answer is seemingly illogical and wrong. But the quantitative approach is useful here because it can find flaws in intuitive thinking. It teaches researchers to not always trust their gut, especially because their gut can be biased and lead to erroneous conclusions. Approaching the Monty Hall problem with the quantitative method, it's easy to see that the correct solution is to always take the third envelope, instead of trusting your first pick. This solution seems wrong, but it's proven time and again quantitatively. This isn't to say that quantitative methodologies are a panacea, merely that quantitative methodologies are capable of finding truth that other approaches fail to find.

% WEAKNESSES, INCLUDING TECHNIQUES
One weakness of the quantitative approach is that even though everything can be measured quantitatively, it is sometimes unethical to take random samples or make random assignments. For example, accumulating enough data to prove a causal relationship between smoking and lung cancer took many, many years because it is unethical to randomly assign people to smoke, and then check if they have cancer. However, over time, enough data was gathered that causality could be shown, even without randomization.

Another weakness, as discussed by Furlong, is confirmatory bias and its associated dangers. Confirmatory bias is when a researcher has a certain notion about the way things work~--- a paradigm through which they see the world~--- which causes them to predominantly see things the way they think they should be, and not how they actually are. Observationally, I've seen this with people's skewed views of the Church: they have a negative attitude about the Church or its members. Everything they see gets filtered through this negative lens, causing them to be further disgusted by the Church; when in reality, it's their confirmatory bias~--- their false paradigm~--- that causes incorrect conclusions.

% APPROPRIATE TIMES TO USE (included in the previous sections as examples (mostly))
Of course, there are times when it isn't appropriate to use quantitative methods. Notable examples are when funding is an issue, when hypotheses are unclear, or when future research avenues are unknown. In these cases, observational studies are great because they can help shine a light on possible research areas, usually with greatly reduced time and money commitments.

A recent example comes from the comedian Aziz Ansari who published \textit{Modern Romance} in June 2015. For years, Aziz had been troubled with the modern dating scene, especially the nonchalant attitude many approached dating with. After years of observational study and no concrete answers, he became so frustrated that he decided to figure out if there was something statistically wrong in the dating world, and if so, what it was. He started with a literature review, but was unable to find anything on this specific topic, so he reached out to a sociology professor and they performed a massive, two year study to assess the problems with modern dating in the United States, France, and Japan. This book covers his initial woes, as well as the qualitative and quantitative studies performed, spanning the past several years. Aziz's example is notable because he started with an observational study, which in turn motivated a literature review, and finally a full-scale investigation to find answers. This is an excellent example of how observational studies can be used to find research areas.

% CONCLUSION
Quantitative research methods are strong because they provide reliable and statistically valuable results. The methods are repeatable and can be proven or disproven by other researchers repeating the same experiment. Over time and through much experimentation, hypotheses turn to theory\footnote{I think it's important to note here that I'm using the scientific definitions behind hypotheses and theories, not what was discussed in class. A hypothesis is simply an idea, but through experimentation, hypotheses turn to theories as more and more results are found that support the original hypothesis. I didn't think this was totally clear in class (and maybe even a little backwards) and wanted to clarify.} as more and more valid results are found supporting them. When experiments are properly set up and randomized, the results can infer causal relationships and relationships to the entire population. These qualities are some of the most important in setting quantitative research methods apart from other research methods, and form the basis for the strength behind quantitative research methods.

\end{document}

Notes from the teach:
Support claims and arguments.  Get rid of clunky sentences.
Have flowing logic - not just filled with all the things you remember. Your paper should be heading somewhere.

- Stick to one position and go on it
- Not a hauppauge of what you remember from the reading
- Everything heading to a conclusion
- Have a flow to it
* If you can't properly defend it - don't put it in


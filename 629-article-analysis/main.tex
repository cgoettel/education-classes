\documentclass[12pt]{article}
\usepackage[margin=1in]{geometry}

\usepackage{setspace}
\doublespacing

\usepackage[explicit]{titlesec}
\titleformat{\section}{}{}{0em}{\textbf{\thesection\ \ #1}}

\title{Article analysis}
\author{Colby Goettel}

\begin{document}
\maketitle

The article I am analyzing is \textit{Predicting Academic Success in Electronics} by Barry M.\ Lunt. Professor Lunt is my thesis committee chair, and this article (adapted from his dissertation) is the basis of my thesis, \textit{A cognitive approach to predicting academic success in computing}.

\section{What phenomenon was the article concerned with? What were the researchers studying, or what were their research questions?}
The article focused on the three electronics fields: electronics technology (ET) (2-year), electronics engineering technology (EET) (4-year), and electrical engineering (EE) (4-year). The Kolb Learning Styles Inventory (LSI) was administered to students in these various programs, and multiple regression models were used to try and find correlations between learning style and academic success in the electronics programs. The research questions are as follows:
\begin{enumerate}
    \item What are the best predictor variables for predicting academic success in electronics?
    \item Is abstract learning preference an effective discriminator between students in the three main types of electronics programs?
    \item What is the best multiple-regression model that can be derived for predicting success in each of the three types of electronics programs?
\end{enumerate}

\section{Did the author justify their study? Did the authors provide rationale as to why the studies should have been conducted? If so, were the rationales persuasive to you? Why or why not?}
The purpose of the research was to determine if there exists a correlation between learning preference and academic success in ET, EET, and EE. If there is a statistically significant correlation, then the LSI could be used as an accurate discriminator to help students choose in which electronics program they should enroll.

This rationale is persuasive to me because it's about the same rationale as I'm using for my thesis: most students don't understand the difference between computer science (CS), information systems (IS), and information technology (IT)\footnote{To be fair, most IT seniors don't know the actual difference.} and struggle to determine which major is right for them. If it were possible to help potential students choose a major by having them take a simple test, that seems like a good thing to figure out.

\section{Did the author employ the most appropriate methods for this study based on the research questions?}
Having done my own literature review on this topic and successfully defended my prospectus, I'm going to say yes. The Kolb LSI is a wonderful, validated tool for determining someone's learning style. The test has been around, validated, updated, and re-validated since 1971. I also strongly agree with the theories behind the test, but you'll just have to wait until my final to read about that.

The multiple regression model that was found accurately finds correlations where they exist. Additionally, it has an appropriate amount of variables which helps justify its validity (1--5 variables makes sense; 30 means it's probably dead wrong). The research also found several areas where correlations do not exist~--- some of note, some of no surprise.

The students were randomly sampled and he had a participation rate of 45\%. Because the students were randomly sampled and because the participation rate was $>10\%$, inferences to the population can be drawn. This is especially important in educational research where students' records are protected by federal law and can only be accessed with a student's permission. It's important, then, to randomly sample (or get the entire population) so that, even though not everyone will respond, inferences can still be drawn. When research isn't done like this, it's possible to form opinions based off the researcher's findings, but impossible to apply these conclusions to other groups. That puts a serious hamper on future work. Thankfully it's not what happened here.

\section{Were the studies insightful and informative? Why or why not?}
My goodness yes. The research was properly set up allowing for further research of the same type in other fields. It drew useful conclusions which helps with future work because it means that work in other fields will probably not be useless.

\section{What kind of claims or conclusions did the author make based on their research? Did the claims seem justified in light of the methods used and data collected? Why or why not?}
The research drew the following conclusions for each research question:
\begin{enumerate}
    \item Best predictor variables are given in a pretty convoluted table. Not going to reproduce it here. Rest assured, it's solid.
    \item Abstract learning preference (AC-CE) was a variable ``in three of the four multiple-regression models developed to answer question \#3, indicating that abstract learning preference \emph{does} make a unique contribution in a multiple-regression model for predicting academic success in electronics.'' The overall conclusion is that AC-CE was a good discriminator for EE and EET students.
    \item The formula is given in the paper. It wouldn't make sense to reproduce it here.
\end{enumerate}

Yes, the claims are justified in light of the methods used. Since this is a quantitative study and the statistical work was clean, the conclusions drawn are solid.

Many implications were drawn from this research. The main implications are that it is possible to predict how well a student will perform in an electronics program based off a number of variables from their high school transcript, that AC-CE is a good discriminator and can be used to help potential electronics students choose a major, and that EE students generally have higher high school grades and ACT scores.

\section{References}
\begin{itemize}
    \item Lunt, Barry M. \textit{Predicting Academic Success in Electronics}, 1996. Journal of Science Education and Technology, volume 5, issue 3, page 235.
\end{itemize}

\end{document}

Notes from the teach:
Support claims and arguments.  Get rid of clunky sentences.
Have flowing logic - not just filled with all the things you remember. Your paper should be heading somewhere.

- Stick to one position and go on it
- Not a hauppauge of what you remember from the reading
- Everything heading to a conclusion
- Have a flow to it
* If you can't properly defend it - don't put it in


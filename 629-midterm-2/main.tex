\documentclass[12pt]{article}
\usepackage[margin=1in]{geometry}

\usepackage{setspace}
\doublespacing

\title{Ethnography}
\author{Colby Goettel}

\begin{document}
\maketitle

% Prompt: Select one qualitative method we've covered this semester and discuss what it is intended to achieve as a form of inquiry. In answering this question, you should be able to describe this research method's purpose, major characteristics, and strength/limitations as an educational research method.

% Teach is looking for: related issues, what it might be used for, good for/bad for, critiques.

% DEFINE
Ethnographic research is a form of qualitative research that originated in anthropology. Compared to quantitative research, it is open-ended and subjective. However, it is an immensely important form of research because it aims to be holistic and provide context through both quantitative and qualitative means. It does this by following the principles of the scientific method. It provides real, valid results.

% PURPOSE
The main purpose behind qualitative research, especially ethnographic research, is to provide context: why things happen, not just hard figures proving they happen to a statistically significant value. Ethnographic research describes the context of a specific people, their culture, how their history affects and effects them. It defines things that might seem hidden, especially to the casual observer.

% MAJOR CHARACTERISTICS
To uncover these hidden things, ethnographic researchers are involved in the lifestyle of those they observe to varying degrees. Some forms of ethnographic research call for passive observation; others call for complete immersion, while allowing for every degree in between. Because it comes from anthropology, it tries to be as non-intrusive and natural as possible. However, every time measurements are taken, the nature of the thing being measured changes. This is especially true of ethnographic research where researchers assimilate themselves into a particular niche for an extended period of time.

% STRENGTHS & LIMITATIONS
Ethnography is incredibly comprehensive, but unfortunately this comes at a price: most observations are not quantifiable, and it can be difficult sometimes to generalize what is being said, simply because of the amount of data being gathered. One of the recent guest lecturers brought up an important point about this: it requires hours upon hours taking notes and performing interviews and transcribing to begin to form a picture of a people. Ethnographic research tries to make sense of the full range of social behavior in a specific setting which is costly, both monetarily and temporally. Because of the massive depth that ethnographic research tries to engulf, it is necessary to limit the scope of the research. It would simply be impossible~--- at least logistically~--- to try and capture everything, so researchers must limit the number of things they try to observe and explain.

The researchers introduce bias to their observations. They must know the language and customs, and their accent or actions could influence people's behavior. If a family is being observed, the children will act differently because a strange adult is in the home. Researchers need to account for how their presence impacts the study. And since the research is subjective, it can be difficult to tell if it's reliable and valid, especially with the amount of bias that can be introduced by poor research methods. But if researchers are careful in their research, means, and documentation to control confounding variables, they can lessen their footprint on the observations and produce valid, reliable research.

One of the biggest limitations of ethnographic research is that it attempts to define other cultures according to existing theories. This unfortunately means that everything gets squeezed into a predefined box and must fit with current definitions; that or new theories must be developed. This doesn't necessarily mean that things are forced into a Western definition, but they \emph{are} forced into existing definitions. Additionally, those reading the studies all come from their own backgrounds and filter the information through their own paradigms. This causes much more ambiguity than in quantitative research where the facts are calculated. However, this limitation's effects can be limited if ethnographers are careful to write things in specific ways and give justifications for why they chose specific theories.

Another big limitation is that researchers must have access to the location. On my mission, we were contacted by a First Nations tribe in Baker Lake, Nunavut\footnote{Fun fact: Baker Lake is the geographical centre of Canada.} who said that if we could get missionaries to them, they would fill an auditorium with people interested in our message. However, in these secluded parts of Canada, if the tribal leaders don't approve of the visitors, they're not allowed access to the tribe's lands. Thankfully, this wasn't a problem, but it presents a clear case where even though the people might be willing, the local government must still give consent for researchers (or missionaries) to enter their land.\footnote{In case you want closure: We ended up not sending missionaries there because it's about 1200~mi north of Winnipeg, accessible only by plane. It would have been impossible to keep the doctrines of the Church pure and we were told by Elder Clayton (of the Seventy) that we should forgo sending missionaries there at this (2007) time.}

% EXAMPLE
Unrelated to this exam, I serendipitously read an interesting ethnographic study\footnote{A secondary source if we're being pedantic.} today about why people in Norway are happy during winter. The researcher received a grant to go to the most northerly parts of Norway for around seven months through the winter and observe, participate in, and interview residents in the remote town. She found that the main reason the Norwegians were happy was because they chose to be. In America, we commiserate together about the cold. We collectively hate it. In Norway, they look forward to ski season; they enjoy being cozy (and not just the `Netflix and chill' variety) and drinking hot chocolate and enjoying each other's company. They have festivals to celebrate the cold. In America, we just bitch and moan about it.

This research is an excellent example of why ethnography is used. The researcher was able to assimilate herself into the local culture, interview, observe, and stay for a sufficiently long time that conclusions could be drawn.

% CONCLUSION
The end of all ethnographic research is to subjectively describe the context of a particular, specific location, and the customs of its peoples and cultures. It looks at the entire picture~--- history, language, culture, everything~--- in order to make informed conclusions about people. Since it uses both quantitative and qualitative means, it is able to not only show that certain behaviors happen, it is able to explain why they happen.

\end{document}

Notes from the teach:
Support claims and arguments.  Get rid of clunky sentences.
Have flowing logic - not just filled with all the things you remember. Your paper should be heading somewhere.

- Stick to one position and go on it
- Not a hauppauge of what you remember from the reading
- Everything heading to a conclusion
- Have a flow to it
* If you can't properly defend it - don't put it in

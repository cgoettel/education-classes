\documentclass[12pt]{article}
\usepackage[margin=1in]{geometry}

\usepackage{setspace}
\doublespacing

\title{The acquisition metaphor (for morning people)}
\author{Colby Goettel}

\begin{document}
\maketitle

% Prompt: Describe in your own words the two metaphors for learning presented in Sfard's article "On Two Metaphors for Learning and The Dangers of Choosing Just One." Be sure to clarify how the two metaphors differ and why that difference is significant. (That is, what's the debate about?) Finally, from your perspective, what is the advantage or disadvantage of having both metaphors (and possibly others) in the discipline of education? Defend your answer.

% DEFINE AM
% Learning
% Concepts build on each other (justified?)
% Fact-based
% Memory. Information is stored (yeah, duh).
The acquisition metaphor (AM) is a convoluted way of saying that people learn by learning (as opposed to doing). Facts, ideas, and concepts build on each other. These pieces of information~--- objects~--- are what is known as knowledge. Knowledge is then internalized and stored in memory. Historically, AM has been the metaphor used for thousands of years to describe knowledge and learning. Generally, when people think about learning, they think AM.

% DEFINE PM
% Doing
% It's not something we have, it's something we do
% social, cultural
% hippie dippie being a part of something
% part of a community (hardcore/punk/ska/skinhead?)
% all-encompassing
The participation metaphor (PM) is a learning style dictated by actions: people learn by doing. PM advocates teach that knowledge is not something that people have, it's societal and communal. Learning is done by being part of a community. This type of learning is all-encompassing~--- learning isn't something that anyone has, it's one part of the whole.

% CONTRAST AM AND PM. What's the debate about?
% AM: mind. facts. knowledge (as generally understood).
% PM: bonds between people and things
% AM: internalization of knowledge
% PM: all-encompassingness of ideas
Proponents of AM and PM have formed into camps, warring with each other about which is the True and Living Metaphor to describe learning. I firmly believe that this so-called debate is artificial, created for the sake of argument. It allows each side to better understand themselves. AM and PM focus on two completely different aspects of learning. They're definitely related, but they also exist independently. These metaphors are symbiotic and thrive on each other.

Knowledge, neurologically speaking, is when the myelin sheaths of neurons form along certain pathways, making memory. These memories, like muscle, are reinforced as they are used and weaken when not.

In another sense, knowledge is only gained by being part of a community. Whether that be the chess club, a math class, or on a national or global level, these are all communities in which individuals learn and form memories.

% DISADVANTAGES OF HAVING BOTH
PM argues that no knowledge is centrally held, but is communal. True, but it's 2015: there is no longer this primal sense of community in the world. Technology has bridged that gap and allowed individuals to join whichever communities they'd like. Go to a library and join the other communities by reading their literature. E-mail a professor and ask questions. It's odd that people are arguing PM in light of all the technological advances that seem to make it obsolete (or, at least, in need of some serious revision).

% ADVANTAGES OF HAVING BOTH
% PM is the stupidest argument I've ever read. No freaking duh you learn in a societal way, BUT THEN YOU INTERNALIZE THOSE CONCEPTS.
% "objectifying knowledge"? Are you kidding me?
In reality, it seems that these metaphors are actually a methodology: first, an individual learns in a community (PM), but then internalizes that knowledge (AM). PM has it wrong that knowledge is some hippie-dippie, nebulous thing that exists but doesn't really exist and was created \textit{ex nihilo}: it's facts and concepts and ideas and strategies and methodologies that are taught to people in a society and are then propagated through language and actions. These knowledge objects are then internalized and stored in the individual's memory (AM) and in the memory of the community (PM).

% CONCLUSION
These metaphors exist in harmony. People have created an artificial debate because it helps them learn. Both metaphors exist. Everyone has experienced both in their lives because that's how the world works.

\clearpage
\begin{center}
    {\LARGE Information processing}
\end{center}
% Prompt: Describe the classic information-processing (i.e., Atkinson and Shiffrin) model. What are its components and how does it work? (You can add updated understandings as well, regarding working memory, forgetting, etc.) In what sense is this a model of human learning? In other words, what does learning involve from this perspective?

\end{document}

Notes from the teach:
Support claims and arguments. Get rid of clunky sentences.
Have flowing logic - not just filled with all the things you remember. Your paper should be heading somewhere.

- Stick to one position and go on it
- Not a hauppauge of what you remember from the reading
- Everything heading to a conclusion
- Have a flow to it
* If you can't properly defend it - don't put it in


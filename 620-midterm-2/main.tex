\documentclass[12pt]{article}
\usepackage[margin=1in]{geometry}

\usepackage{setspace}
\doublespacing

\title{Vygotsky on learning and development}
\author{Colby Goettel}

\begin{document}
\maketitle

% Prompt: Describe the relationship between learning and development according to Vygotsky's theorizing. Be sure to define key concepts and show how they relate to one another. Also, indicate how Vygotsky's view differs from other prominent positions on this topic.

% DEFINE
According to Vygotsky, learning is not something we have, it's something we do. Learning is participatory. It is experiencing something that happened in the world. This is incredibly close to how David Kolb defined learning: ``[T]he process whereby knowledge is created through the transformation of experience'' (Kolb, 2014).

In contrast to learning, development is the actual progress being made by the learner. Development is the fruits of learning. In Vygotsky's mind, there are two types of development: actual development and proximal development. Actual development is what a learner can do. Proximal development, however, is the distance between what a learner can do on their own versus what they can do with someone more knowledgeable than them. This more knowledgeable person is not necessarily a master of their domain, but a peer or a teacher~--- they just have to be someone more knowledgeable.

% RELATIONSHIP BETWEEN LEARNING AND DEVELOPMENT
The relationship between learning and development is key to understanding how to teach. Since learning is based on experiences had in the world, all teaching must have experiential qualities. And since development is the actual progress being made by the learner, it is key to track progress and experiences so that development can be assessed.

% OTHER VIEWS
Early theories from Piaget and others taught that development comes before learning, the exact opposite of Vygotsky's theory. This early theory taught that each learner was at a different point in their understanding and that they should be taught accordingly: teachers should teach according to how ready each learner is. Piaget taught that there are developmental limitations, that learners could only learn up to a certain point, depending on the learner's age and biological progression, among other factors. The gist being that a teacher can't teach someone something they're not ready for.

James later theorized that learning \emph{is} development. This behaviorist theory taught that everything in the environment shapes the learner. It's relatively close to Vygotsky's theory, especially as it relates to learning from the environment, but differs in that is equates learning and development; Vygotsky clearly argued that learning comes before and causes development.

The last theory before Vygotsky comes from Koffka and Gestalt who taught that learning and development constantly effect and inform each other. The only difference between this theory and Vygotsky's theory is semantic. Arguably, Vygotsky would agree that learning and development influence each other, he was just clear in stating that learning precedes development.

% HOW VYGOTSKY DIFFERS
Vygotsky theorized that learning is experiences had in the world. That learning happens all the time. In fact, everyone in the world that directly or indirectly influences the learner contributes to learning. This idea of communal learning, influenced by Marxism, is the idea that learners interact with others who already know how to live in the society; and learners learn how to live in their society through these interactions. Communal learning makes up a huge part of Vygotsky's idea of cultural apprenticeship, apprenticeship having more to do with interacting than coaching.

Because Vygotsky came from a communist nation, his thinking is heavily influenced by Marxist ideas. A prevalent, Marxist idea at his time was called Sovietization: if you can make someone live like a communist, they will think like a communist. This worked right into Vygotsky's collectivist thinking and influenced his ideas on social reproduction and enculturation: we become like the community when we learn to think and act like them. This cultural learning is not restricted to schooling because tons of learning happens outside of school and especially before formal schooling begins. Children begin learning long before school: learning about their culture, their surroundings, their family.

Vygotsky thought of learning as a verb, not a noun: learning is not necessarily something that is had, it's something that is done. This works perfectly into the participation metaphor. He theorized that development is what a child is capable of doing: development is the fruit of learning. A great line in one of his papers is (paraphrasing) that what a child can do with others is more indicative of their development than what they can do alone.

% CONCLUSION
Vygotsky was so radically different than his predecessors that he started an entire movement. The idea that learning is what influences development was in stark contrast to previous theorists who either believed that development came first or that learning and development were hopelessly intertwined. Vygotsky changed the way we think about learning.

\clearpage
\begin{center}
    {\LARGE The narrative approach}
\end{center}

% Prompt: Describe a narrative approach to instruction (and learning), as if discussing it with someone unfamiliar with the idea. Then offer a critique. What might be the most significant advantage of this approach? What might be its most significant disadvantage? Why?

The narrative approach to instruction is when the learning process is reinforced with stories. People's entire lives are stories, so they're used to hearing and interpreting stories. When a difficult subject is being taught, a story can help the learner to understand what's being taught. Stories are also great for teaching life principles; hence why Jesus so often taught in parables.

One of the great things about teaching through stories is that the story can take on new meaning throughout the learner's life. A great example is the scriptures. When someone is reading the scriptures for the first time, they're constantly focusing on what is literally happening, rarely seeing any deeper meaning. However, once they have read through a story a few times, the story begins to take on new meaning. This is especially true for learners as they age: stories mean new things when someone is about to get married or have a kid or when they're struggling with a loved one's death.

A common way of using the narrative approach is to tell stories during a lecture. These stories can teach great principles, get the students' attention, and help students remember what was taught. A few weeks ago, I gave a talk in sacrament focused on the temple. I'm one of the temple co-chairs and we'd spent many weeks trying to figure out why our ward's temple attendance was so low. The thing that I kept coming back to is that it was a priority problem. So I shared a story that Elder Didier of the Seventy shared many years previous:
\begin{quote}
    Many years ago, a man was traveling and went to church. When he walked in, he noticed that there was a clock really high up the wall. Not, like, a little bit up the wall~--- way up the wall. So he asked one of the church members why the clock was so high up the wall. The man said, ``It didn't used to be like that; in fact, the clock was actually much lower on the wall. But when people came to church they would see the time on the clock, check the time on their watch, and then fix the time on the clock. Then another person would come to church, see that the time on the clock, check their watch, and fix the time on the clock. So we moved the clock really high up on the wall. And an interesting thing happened: people would come to church, see the time on the clock, check their watch, realize that they were wrong, and fix the time on their watch.''
\end{quote}
This story is great because it's easy to remember and teaches a powerful principle, one that doesn't even need to be explicitly stated: when we come to church, we don't change the time on the wall, we set our watches~--- and our lives~--- according to the time on the wall. It's easy to understand that the clock on the wall is symbolic of the Lord's will for us. We do not come to church and complain about something not being PC or in vogue; we come to church and set our lives in the order the Lord has instructed.

The most significant advantage to approaching education with a narrative approach is that students remember what was taught. And if the stories are good enough, they'll pass those stories along to others they come in contact with. I heard the clock story a couple months ago when I was passing through St.\ George driving my mom to California. I've since shared the story with multiple friends, family, my entire ward, and now my professor. Because the story was enticing, easy to remember, and taught a great principle, I have been able to share it with many others and bless their lives because of it.\footnote{This hearkens to my course project discussing that the purpose of all learning is to uplift mankind.}

However, it's not all sunshine and puppies with the narrative approach. The hardest part of the narrative approach is coming up with applicable stories. It's time consuming. It's hard. And sometimes, the stories that work so well for one person never seem to work for others. Another huge problem is that stories are huge. They're time consuming. Some stories require so much background information that it's not practical to share them.

Another problem is that sometimes students don't relate to the story. When I was on my mission, we had to go to the General Relief Society Meeting to translate for one of our recent converts. Every talk started off ``Sisters\dots'' and I immediately stopped paying attention because I knew it wasn't for me. And those weren't even stories! They were just talks! It's like using a story from the ghettos of Brazil to teach a bunch of rich, entitled high school students in Orange County some life lesson. It's not going to work. They have no idea what that life is like. I know, I grew up there.

Thankfully, there are so many stories available that once a teacher has gained enough life experience, they will have a plethora of good stories to choose from. The more a teacher uses stories, the faster they will be able to find the right story for a specific lesson. And by using stories, the teacher will be able to teach difficult or obscure principles in a way that can impact students' lives now and for years to come.

\end{document}

\documentclass[12pt]{article}
\usepackage[margin=1in]{geometry}

\usepackage{setspace}
\doublespacing

\title{Vygotsky, the absolute madman}
\author{Colby Goettel}

\begin{document}
\maketitle

% Prompt: Describe the relationship between learning and development according to Vygotsky's theorizing. Be sure to define key concepts and show how they relate to one another. Also, indicate how Vygotsky's view differs from other prominent positions on this topic.

% DEFINE
According to Vygotsky, learning is not something we have, it's something we do. Learning is participatory. It is experiencing something that happened in the world. This is incredibly close to how David Kolb defined learning: ``[T]he process whereby knowledge is created through the transformation of experience'' (Kolb, 2014).

In contrast to learning, development is the actual progress being made by the learner. Development is the fruits of learning. In Vygotsky's mind, there are two types of development: actual development and proximal development. Actual development is what a learner can do. Proximal development, however, is the distance between what a learner can do on their own versus what they can do with someone more knowledgeable than them. This more knowledgeable person is not necessarily a master of their domain, but a peer or a teacher~--- they just have to be someone more knowledgeable.

% RELATIONSHIP BETWEEN LEARNING AND DEVELOPMENT
The relationship between learning and development is key to understanding how to teach. Since learning is based on experiences had in the world, all teaching must have experiential qualities. And since development is the actual progress being made by the learner, it is key to track progress and experiences so that development can be assessed.

% OTHER VIEWS
Early theories from Piaget and others taught that development comes before learning, the exact opposite of Vygotsky's theory. This early theory taught that each learner was at a different point in their understanding and that they should be taught accordingly: teachers should teach according to how ready each learner is. Piaget taught that there are developmental limitations, that learners could only learn up to a certain point, depending on the learner's age and biological progression, among other factors. The gist being that a teacher can't teach someone something they're not ready for.

James later theorized that learning \emph{is} development. This behaviorist theory taught that everything in the environment shapes the learner. It's relatively close to Vygotsky's theory, especially as it relates to learning from the environment, but differs in that is equates learning and development; Vygotsky clearly argued that learning comes before and causes development.

The last theory before Vygotsky comes from Koffka and Gestalt who taught that learning and development constantly effect and inform each other. The only difference between this theory and Vygotsky's theory is semantic. Arguably, Vygotsky would agree that learning and development influence each other, he was just clear in stating that learning precedes development.

% HOW VYGOTSKY DIFFERS
Vygotsky theorized that learning is experiences had in the world. That learning happens all the time. In fact, everyone in the world that directly or indirectly influences the learner contributes to learning. This idea of communal learning, influenced by Marxism, is the idea that learners interact with others who already know how to live in the society; and learners learn how to live in their society through these interactions. Communal learning makes up a huge part of Vygotsky's idea of cultural apprenticeship, apprenticeship having more to do with interacting than coaching.

Because Vygotsky came from a communist nation, his thinking is heavily influenced by Marxist ideas. A prevalent, Marxist idea at his time was called Sovietization: if you can make someone live like a communist, they will think like a communist. This worked right into Vygotsky's collectivist thinking and influenced his ideas on social reproduction and enculturation: we become like the community when we learn to think and act like them. This cultural learning is not restricted to schooling because tons of learning happens outside of school and especially before formal schooling begins. Children begin learning long before school: learning about their culture, their surroundings, their family.

Vygotsky thought of learning as a verb, not a noun: learning is not necessarily something that is had, it's something that is done. This works perfectly into the participation metaphor. He theorized that development is what a child is capable of doing: development is the fruit of learning. A great line in one of his papers is (paraphrasing) that what a child can do with others is more indicative of their development than what they can do alone.

% CONCLUSION
Vygotsky was so radically different than his predecessors that he started an entire movement. The idea that learning is what influences development was in stark contrast to previous theorists who either believed that development came first or that learning and development were hopelessly intertwined. Vygotsky changed the way we think about learning.

\clearpage
\begin{center}
    {\LARGE }
\end{center}
% Prompt: Describe a narrative approach to instruction (and learning), as if discussing it with someone unfamiliar with the idea. Then offer a critique. What might be the most significant advantage of this approach? What might be its most significant disadvantage? Why?

p. 66 Clark article- Narrative Learning in Adulthood

Narrative overall - The Narrative approach to instruction is basically when you introduce story elements into the learning process. People have a tendency to work within the realms of types of stories they have heard and are familiar with but there are countless ways of remaking those stories into something new and different -Chomsky language stuff

Describe Narrative (story) - One way you can do this is by telling stories during the lesson to teach principles, encourage cognitive processes, or garner interest.

Story critique - Sometimes learners don’t relate to the story or the protagonists in the story so they won’t learn anything from the process. Also this is hard to teach because it takes a lot of effort to come up with relevant stories and take the time to use them, etc. 

Describe Narrative (protagonist) - Another way of doing this is by making the learning process itself into a story this can be done through case study like things or by encouraging the learner to participate in the “story” as their own protagonist. Which means they have to see themselves as the hero and the learning, or the projects, or the tests as the conflict and then they learn and overcome by their hard work and diligence to follow through and rise triumphant

Protagonist critique - This is really hard to encourage people to see the world this way if they don’t already. Honestly I don’t know how you would do it. Plus just super time and intensive to organize classes like this

\end{document}

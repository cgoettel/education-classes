\documentclass[man,natbib]{apa6}

\title{Thoughts and stance}
\shorttitle{Thoughts and stance}
\author{Colby Goettel}
\affiliation{Brigham Young University}

\abstract{TODO}

\begin{document}
\maketitle

% Critical reflection paper on quantitative, qualitative, mixed-method, and design-based approached to educational research.
% This should be a summary of your thoughts regarding the strengths, weaknesses, value, and limitations of each approach in relation to the others.
% But, most importantly, it should indicate your stance on how to best conduct inquiry in your topic area.
% Your paper should deal with substantial issues pertaining to method use and inquiry, making reference to ideas and practices we have covered this semester.

% Show thinking and position-taking.
% Thoughts on the relative value of these approaches, perhaps with respect to different projects or questions.

% HOW DO THEY RELATE?

Not a lot done in IT so quantitative makes more sense. Once more has been done, then qualitative and mixed methods can be used. There's been a lot done in computer science, but information technology is a more recent field and a foundational body of research is almost finished being put in place, but we still need more. And we're all super technical people so it makes more sense for us to work quantitatively. Now, I understand that qualitative research is different and hits different things, but we're generally not interested in those things. Most of our research is into hard facts about how students are performing, not about feelings and motivations; although those areas definitely need to be researched.

Acceptance of Kolb: http://www.whitewater-rescue.com/support/pagepics/lsitechmanual.pdf

I don't freaking care which is best in my field because there's no right answer. Every study is different. I have no prejudice when it comes to reading an article as long as they PROVE THEIR POINT. Pick whichever method is going to best prove your point, and then do it.

I will say, though, that quantitative makes the most sense to my mind. Not necessarily the most sense in IT, but definitely to my mind. When I read a qualitative study I always want to know \emph{exactly} how accurate and true it is. I want $p$-values!

\end{document}

\documentclass[man,natbib]{apa6}

\title{Critical reflection paper}
\shorttitle{Critical reflection}
\author{Colby Goettel}
\affiliation{Brigham Young University}

\abstract{Various methodologies can be used when performing research. However, some of these methods are expensive or time-consuming, and many don't work for some forms of research. This paper looks into quantitative, qualitative, mixed-method, and design-based research arguing the pros and cons of each. Finally, it declares a stance on which methodology is best suited for research in information technology education.}

\begin{document}
\maketitle

% Your paper should deal with substantial issues pertaining to method use and inquiry, making reference to ideas and practices we have covered this semester.
% Show thinking and position-taking.
% Thoughts on the relative value of these approaches, perhaps with respect to different projects or questions.

% Critical reflection paper on quantitative, qualitative, mixed-method, and design-based approached to educational research.
\section{Methodologies}


\subsection{Quantitative}
% STRENGTHS and VALUE
Quantitative research is a hard, scientific methodology that seeks to find empirical results by focusing on data collection and analysis. It follows the scientific method in order to observe the world and then uses mathematical and statistical techniques to analyze the results. Its strength lies in the fact that it provides hard evidence to support claims.

When test subjects are randomly selected from the population, it's possible to infer that the results are representative of the population. When the treatments are randomly assigned, causal relationships can be inferred. These characteristics compound and lend power to quantitative studies.

Another great strength of quantitative research is that it is systematic. When performing experiments, researchers can go through related variables, in order, and systematically check things off as they relate or don't relate. This principle allows researchers to remove bias and confounding variables. Unfortunately, bias can be found in every form of inquiry and researchers must be careful to address it in their studies. For example, confirmatory bias \citep{furlong2000research} is when a researcher has a certain notion about the way things work~--- a paradigm through which they see the world~--- which causes them to predominantly see things the way they think they should be, and not how they actually are. As long as researchers address their biases, other researchers know on which sections to tread carefully.

% WEAKNESSES
One weakness of the quantitative approach is that even though everything can be measured quantitatively, it is sometimes unethical to take random samples or make random assignments. For example, accumulating enough data to prove a causal relationship between smoking and lung cancer took many, many years because it is unethical to randomly assign people to smoke, and then check if they have cancer. However, over time, enough data was gathered that causality could be shown, even without randomization.

% LIMITATIONS
A slight limitation of quantitative research, like other methodologies, is the struggle to gain access to people and their information. This is especially true for academic research where researchers must gain access to students' academic records which are protected by federal law. However, there are Research Advisory Boards that can help to overcome this hurdle, allowing research to move forward in an ethical manner that doesn't adversely affect people, their information, and their security.

There are, of course, times when it isn't appropriate to use quantitative methods. Notable examples are when funding is an issue, when hypotheses are unclear, or when future research avenues are unknown. In these cases, alternate methodologies are useful because they can help to shine a light on possible research areas, usually with greatly reduced time and money commitments.

Another limitation is that although quantitative research excellently shows how strongly things happen, it fails to give any explanation as to why things happen. Researchers who are looking for answers about peoples' motivations would do well to look into the other forms of inquiry.

\subsection{Qualitative}
% STRENGTHS
One of the biggest strengths of qualitative research is the varied amount of approaches in which it can be performed. Some of the most common approaches are narrative, phenomenology, grounded theory, ethnography, and case study. These stories focus on people, their stories, what motivates them. And they work for greatly varied units of analysis, from individuals to entire cultures.

Qualitative research is great because, unlike quantitative research that gives exact figures, it explains \emph{why} things happen. Its history is rooted in anthropology, sociology, and the other social sciences which pedigree helps it describe people. Unlike other methodologies, qualitative research focuses on people as people instead of taking a rigid, scientific approach to analyzing such complex systems.

% WEAKNESSES
However, all this focus on people means that interviews must be performed. These interviews must then be painstakingly transcribed. These processes are heavily time-consuming and can be a major turn-off to performing this type of research. But it all comes down to what the researcher is looking to prove: are they trying to find exact figures and correlations, or are they trying to see why things happen a certain way? If they want exact figures, they should look into different methodologies; but if they want to know why things happen, qualitative research is the way to go because it provides great insights into the whys of life.

The biggest weakness of qualitative research is that it begs the question: how true is this research? Because it is not founded in empirical data, there are no data points to show correlation or causality. Of course, if the sample sizes are large enough, some inferences can still be made, but this is generally not the focus of qualitative research.

% VALUE and LIMITATIONS
Qualitative research is great because it can be a cheap way to get some preliminary research done while funds are being secured for additional, generally more empirical studies. It is definitely a great methodology, but in some fields it's mostly a stepping-stone leading to more research.

\subsection{Mixed-method}
% STRENGTHS and VALUE and WEAKNESSES and LIMITATIONS
Mixed-method research is the union of quantitative and qualitative research. Not only does it provide specific, empirical findings, but also tells the whys of the research. However, this form of research can be long and arduous and not necessary for most studies. It sheds more light on things because it's more involved, but most studies simply don't need the depth that the mixed-method methodology provides, nor can they always afford to perform so much research at once.

\subsection{Design-based}
Design-based research is an iterative approach to research that allows researchers to design specific experiments for their target. While the research is ongoing, the researchers can iterate on the experiment in order to really dig into a specific issue.

% STRENGTHS, WEAKNESSES, and LIMITATIONS
While design-based research is great because it's able to really get into the specifics, it takes a lot of work to design new experiments, get funding, perform the research, iterate, keep researching, iterate more, analyze\dots\ you get the point. Design-based research takes, roughly, a metric buttload of work to design and carry out. And once it's performed, it's not as repeatable for other applications as quantitative and qualitative research. Additionally, every time a new experiment is designed, it has to be validated.

Another common argument against design-based research is that it's not scientific \citep{shavelson2003science}; it doesn't take as much care to isolate variables. And the variables can change during the research which raises more concerns about controls and isolation. Many researchers aren't comfortable with this because it seems like bad design, especially for something called \emph{design}-based research.

% WEAKNESSES and VALUE
Finally, design-based research is quasi-experimental, meaning that subjects are not randomly assigned to control and test groups. This goes against some of the basic tenets of quantitative research and greatly hurts a researcher's ability to make correct inferences about their research. However, this quasi-experimental nature has been defended because it's still useful for developing new theories and hypotheses \citep{brown1992design}. It's meant to be used only for exploratory research.

% But, most importantly, it should indicate your stance on how to best conduct inquiry in your topic area.
\section{My stance}
I honestly don't care which methodology is best in my field because there's no right answer. Every study is different. I have no prejudice when it comes to reading an article as long as they prove their point. Pick whichever method is going to best prove your point, and then do it. However, quantitative research makes the most sense in technology because regardless of the study, I always want to know \emph{exactly} how accurate and true it is. I want $p$-values! I want correlation equations.

My literature review for my prospectus consisted of twenty-two academic articles and it was, except for one article, entirely quantitative. Not a single qualitative or mixed-method research. The only non-quantitative article was a design-based article that iterated through various questionnaires as it developed a validated test for assessing academic satisfaction \citep{nauta2007assessing}.

So while I'm not going to say that quantitative is the end-all, be-all of research, it's what exists in information technology eduction research because technical people want firm figures. And honestly, not that much research has been done in IT so quantitative makes more sense to help establish a foundational body of research. And while qualitative and design-based research are different and hit different aspects, we're generally not interested in those things. Most of our research is into hard facts about how students are performing, not about feelings and motivations; although those areas definitely need to be researched.

And when they are researched, it should be done quantitatively.

\bibliography{references}

\end{document}

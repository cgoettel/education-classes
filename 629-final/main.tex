\documentclass[man,natbib]{apa6}

\title{Critical reflection paper}
\shorttitle{Critical reflection}
\author{Colby Goettel}
\affiliation{Brigham Young University}

\abstract{Various methodologies can be used when performing research. However, some of these methods are expensive or time-consuming and not needed for many forms of research. This paper looks into quantitative, qualitative, mixed-method, and design-based research arguing the pros and cons of each. Finally, it declares a stance on which methodology is best suited for research in information technology education.}

\begin{document}
\maketitle

% Your paper should deal with substantial issues pertaining to method use and inquiry, making reference to ideas and practices we have covered this semester.
% Show thinking and position-taking.
% Thoughts on the relative value of these approaches, perhaps with respect to different projects or questions.
There are many methodologies 

% Critical reflection paper on quantitative, qualitative, mixed-method, and design-based approached to educational research.
\section{Methodologies}
% LIMITATIONS
All the following methodologies share certain weaknesses, like struggling to gain access to people and their information. This is especially true for academic research that must gain access to students' academic records because they're protected by federal law.

\subsection{Quantitative}
% STRENGTHS and VALUE
Quantitative research is a hard, scientific methodology that seeks to find empirical results by focusing on data collection and analysis. It follows the scientific method in order to observe the world and then uses mathematical and statistical techniques to analyze the results. Its strength lies in the fact that it provides hard evidence to support claims.

When test subjects are randomly selected from the population, it's possible to infer that the results are representative of the population. When the treatments are randomly assigned, causal relationships can be inferred. These characteristics compound and lend power to quantitative studies: when test subjects are randomly selected and treatments are randomly assigned, inferences to populations \emph{and} causal relationships can be inferred.

Another great strength of quantitative research is that it is systematic. When performing experiments, researchers can go through related variables, in order, and systematically check things off as they relate or don't relate. This principle allows researchers to remove bias and confounding variables.

% WEAKNESSES
One weakness of the quantitative approach is that even though everything can be measured quantitatively, it is sometimes unethical to take random samples or make random assignments. For example, accumulating enough data to prove a causal relationship between smoking and lung cancer took many, many years because it is unethical to randomly assign people to smoke, and then check if they have cancer. However, over time, enough data was gathered that causality could be shown, even without randomization.

gives a great ``how much'' but no ``why''

% LIMITATIONS
Of course, there are times when it isn't appropriate to use quantitative methods. Notable examples are when funding is an issue, when hypotheses are unclear, or when future research avenues are unknown. In these cases, alternate methodologies are great because they can help to shine a light on possible research areas, usually with greatly reduced time and money commitments.

\subsection{Qualitative}
% STRENGTHS
One of the biggest strengths of qualitative research is the varied amount of approaches to performing qualitative research. Some of the most common approaches are narrative, phenomenology, grounded theory, ethnography, and case study. These stories focus on people, what motivates them, their story. And they work for greatly varied units of analysis, from individuals to entire cultures.

Qualitative research is great because unlike quantitative research that gives exact figures, qualitative research tells \emph{why} things happen. Its history is rooted in anthropology, sociology, and the other social sciences in order to describe people. Unlike other methodologies, qualitative research focuses on people as people instead of taking a rigid, scientific approach to analyzing such complex systems.

% WEAKNESSES
However, all of this focus on people means that lots of interviews must be performed. Those interviews must then be painstakingly transcribed. These processes are heavily time-consuming and can be a major turn-off to performing this type of research. But it all comes down to what the researcher is looking to prove: are they trying to find exact figures and correlations, or are they trying to see why things happen a certain way? If they want exact figures, they should look into different methodologies; but if they want to show why things happen, qualitative research is the way to go because it provides great insights into the whys of life.

The biggest weakness of qualitative research is that it begs the question: how true is this report? Because it is not founded in empirical data, there are no data points to show exactly how much things happened. Of course, if the sample sizes are large enough, some inferences can still be made, but this is not the focus of qualitative research.
% fix conclusion to this paragraph. It's weak.

Another weakness of both quantitative and qualitative research is confirmatory bias \citep{furlong2000research}. Confirmatory bias is when a researcher has a certain notion about the way things work~--- a paradigm through which they see the world~--- which causes them to predominantly see things the way they think they should be, and not how they actually are.

% VALUE and LIMITATIONS
Qualitative research is great because it can be a cheap way to get some preliminary research done while funds are being secured for more empirical studies. It is definitely a great methodology, but it many fields it's a great stepping stone that can lead to better and more research.

\subsection{Mixed-method}
% STRENGTHS and VALUE and WEAKNESSES and LIMITATIONS
Mixed-method research is the union of quantitative and qualitative research. Not only does it provide specific, empirical findings, but also tells the whys of the research. However, this form of research can be long and arduous and not necessary for most studies. It sheds more light on things because it's more involved, but most studies simply don't need the depth that the mixed-method methodology provides, nor can they always afford to perform so much research at once.

\subsection{Design-based}
% STRENGTHS
Design-based research is an iterative approach to research that allows researchers to design specific experiments for their target. While the research is ongoing, the researchers can iterate on the experiment in order to really fig into a specific issue.

% WEAKNESSES and LIMITATIONS
While design-based research is great because it's able to really get into the specifics, it takes a lot of work to design new experiments, get funding, perform the research, iterate, keep researching, iterate more, analyze\dots\ you get the point. Design-based research takes roughly a metric buttload of work to design and carry out. And once it's performed, it's not as repeatable for other applications as quantitative and qualitative research. And every time a new experiment is designed, it has to be validated.

Another common argument against design-based research \citep{shavelson2003science} is that it's not scientific. It doesn't take as much heed to isolate variables. And the variables can change during the research which raises more concerns about controls and isolation. Many researchers aren't comfortable with this because it seems like bad design, especially for something called \emph{design}-based research.

% WEAKNESSES and VALUE
Finally, design-based research is a quasi-experimental form of research meaning that subjects are not randomly assigned to control and test groups. This goes against some of the basic tenets of quantitative research and greatly hurts a researcher's ability to make correct inferences about their research. However, this quasi-experimental nature has been defended \citep{brown1992design} because it's still useful for developing new theories and hypotheses because it's meant to be used only for exploratory research, not for actual, social science research.

% But, most importantly, it should indicate your stance on how to best conduct inquiry in your topic area.
\section{My stance}
I don't freaking care which is best in my field because there's no right answer. Every study is different. I have no prejudice when it comes to reading an article as long as they PROVE THEIR POINT. Pick whichever method is going to best prove your point, and then do it.

I will say, though, that quantitative makes the most sense to my mind. Not necessarily the most sense in IT, but definitely to my mind. When I read a qualitative study I always want to know \emph{exactly} how accurate and true it is. I want $p$-values!

And my entire literature review for my prospectus consisting of twenty-two academic articles was, except for one article, entirely quantitative. Not a single qualitative or mixed-method research. The only non-quantitative article was a design-based article that iterated through various questionnaires as it developed a validated test for assessing academic satisfaction \citep{nauta2007assessing}.

% fix this
Not a lot done in IT so quantitative makes more sense. Once more has been done, then qualitative and mixed methods can be used. There's been a lot done in computer science, but information technology is a more recent field and a foundational body of research is almost finished being put in place, but we still need more. And we're all super technical people so it makes more sense for us to work quantitatively. Now, I understand that qualitative research is different and hits different things, but we're generally not interested in those things. Most of our research is into hard facts about how students are performing, not about feelings and motivations; although those areas definitely need to be researched. But really, when they are researched, it should be done quantitatively.

\bibliography{references}

\end{document}
